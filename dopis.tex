\documentclass[11pt, fleqn]{article}
\usepackage{graphicx}
\usepackage{a4wide}
\usepackage{float}
\usepackage[top=0cm,bottom=0cm,left=1cm,right=1cm]{geometry}
\usepackage{algorithm}
\usepackage{listings}
\pagenumbering{gobble}

%kodovani vstupu
\usepackage[utf8]{inputenc}%latin2,windows-1250
%podpora pro specificke znaky, napr. ceske
\usepackage[T1]{fontenc}
%podpora pro ceske deleni slov atd.
\usepackage[czech]{babel}

%dalsi moznosti zarovnani sloupcu v tabulkach
\usepackage{array}
%vkladani obrazku
\usepackage{graphicx}
\usepackage{wrapfig}
\usepackage{subfig}
\usepackage{fancyhdr}

\title{Roverské zkoušky skautského střediska Athabaska}
\author{}
\date{}

\begin{document}

\maketitle{}
\vspace{-20pt}

\begin{wrapfigure}{r}{0.2\textwidth}
  \vspace{-30pt}
  \begin{center}
    \includegraphics[width=0.18\textwidth]{logo.png}
  \end{center}
  \vspace{-20pt}
\end{wrapfigure}

Ahoj, jestli čteš tuto zprávu, tvoje cesta za roveringem právě začala.
Z následujících výzev by sis měl vybrat tři takové, které tě někam ať už duševně nebo fyzicky posunou. Složitost úkolů je na tobě samotném a většinou ti do nich nikdo nebude mluvit. To, co je pro tebe složité, může být pro jiné triviální. Nenech se tím odradit - jde hlavně o své vlastní překonání a tvůj dobrý pocit.

Stále nemáš vyhráno. Kromě tří výzev musíš také přesvědčit ostaní rovery, že se mezi ně hodíš a že si místo v roveském kmeni zasloužíš. Momentálně ti věří, ale svoji pozici si můžeš v příštích měsících stále pokazit!

\vspace{20pt}
\section*{Cesty roverství}
  \subsection*{Cesta za Sluncem}
  Při této cestě budeš muset vyrazit na celodenní cestu za sluncem. Na cestě tě budou provázet hlavně tvoje myšlenky a otázky na zamyšlení, které ti předají ostatní roveři a rangers. O ty si budeš muset pár dní přem říct. Zamyšlení je hlavní cíl této zkoušky.
  \subsection*{Změna}
  Zkus alespoň na čtrnáctní dní změnit svůj životní styl odepřením některých horších zvyků a naopak přibráním takových závazků, které se pozitivně odrazí na tvém životním stylu. Volba je na tobě, nicméně odepření by tě mělo někam posunout a na dva týdny změnit tvůj všední den. Ideálně ti to nakonec zůstane.
  \subsection*{Bdění}
  Dokážeš ležet v noci hodinu na posteli a neusnout? Jestliže si vybereš tuto zkoušku, probudíme tě v noci a po opláchnutí obličeje budeš muset opět zalehnout do spacáku a bez jakékoliv rozptýlení zůstat ještě hodinu vzhůru. Jedinou společností ti bude noční příroda.
  \subsection*{Obohacení kmene nebo střediska}
  Pomoc našemu roverskému kmeni nebo rovnou celému středisku ať už nějakým velikým výrobkem, nebo třeba opravou něčeho důležitého. Vedení veliké akce, jako je například tábor nebo veliký výlet se také počítá.
  \subsection*{Veliká hra}
  Máš rád přípravu programu? Tato výzva je možná pro tebe. Přichystej pro ostatní rovery a rangers celodenní nebo noční zážitkovou hru. Můžeš to splnit buť na táboře nebo třeba na speciálním výletě. Kromě tvého vlastního subjektivního hodnocení budou hodnotit i ostatní.
  \subsection*{Fyzická výzva}
  Fyzické překonání v pravém slova smyslu. Najdi si nějakou akci, při které ze sebe budeš muset vydat všechno, co máš. Dorbý je například dosti dlouhý \textit{Klábosilák}, závod \textit{Přes 3 vrchy} a nebo zimní tábor (většinou není moc dlouhý, ale výzva to určitě je) 

\end{document}